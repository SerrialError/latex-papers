\documentclass{article}
\usepackage{amsmath}
\usepackage{graphicx}
\usepackage{physics}
\usepackage{geometry}
\geometry{margin=1in}

\title{Comparison of Drive Systems\\Velocity and Force Derivations}
\author{SerrialError}
\date{}

\begin{document}

\maketitle

\section*{1. Coordinate Definitions}

Let:
\begin{itemize}
    \item \( v_x \): Robot velocity in x (strafe)
    \item \( v_y \): Robot velocity in y (forward)
    \item \( \omega \): Angular velocity (yaw)
    \item \( v_{\text{wheel}} \): Wheel tangential speed
    \item \( F_{\text{wheel}} \): Force applied at wheel-ground contact
\end{itemize}

Assume:
\begin{itemize}
    \item Ideal conditions: no slippage or losses
    \item All wheels provide the same torque \( \tau \) and spin with speed \( v \)
\end{itemize}



\section*{2. X-Drive Derivation}

\subsection*{2.1 Wheel Velocity Vectors}

Each wheel is mounted at a 44° angle. Thus:

\[
\vec{v}_{\text{wheel}} = \frac{v}{\sqrt{2}} \cdot
\begin{bmatrix}
\pm 1 \\
1
\end{bmatrix}
\]

\subsection*{2.2 Net Robot Velocity}

To move forward:

\[
v_y = 2 \cdot \frac{v}{\sqrt{2}} = \sqrt{2} \cdot v
\]

Similarly for strafe velocity,

\[
v_x = 2 \cdot \frac{v}{\sqrt{2}} = \sqrt{2} \cdot v
\]

assuming wheels configured to strafe.

\subsection*{2.3 Force Projection}

\[
F_y = 2 \cdot \left( \frac{F}{\sqrt{2}} \right) = \sqrt{2} \cdot F
\quad , \quad
F_x = \sqrt{2} \cdot F
\]


\section*{3. Mecanum Drive Derivation}

\subsection*{3.1 Contact Velocity via Rollers}

Roller directs velocity at 44°:

\[
\vec{v}_{\text{contact}} = \frac{v}{\sqrt{2}} \cdot
\begin{bmatrix}
\pm 1 \\
1
\end{bmatrix}
\]

\subsection*{3.2 Force Perpendicular to Roller}

\[
F_{\perp} = F \cdot \cos(44^\circ) = \frac{F}{\sqrt{2}}, \quad
F_y = \frac{F_{\perp}}{\sqrt{2}} = \frac{F}{2}
\]

\subsection*{3.3 Net Force and Velocity}

Total force and velocity contributions from 4 wheels:

\[
F_{\text{robot}} = 4 \cdot \frac{F}{2} = 2F \quad \text{(total)} \quad , \quad
v_y^{\text{robot}} = 4 \cdot \frac{v}{2} = 2v \quad \text{(total)}
\]

Assuming symmetrical contributions to \( v_x \), similarly,

\[
v_x^{\text{robot}} = 2v \quad , \quad F_x^{\text{robot}} = 2F
\]



\section*{4. Tank (Differential) Drive Derivation}

\subsection*{4.1 Geometry}

Two wheels: Left (L) and Right (R), separated by width \( d \)

\subsection*{4.2 Velocity Equations}

\[
v_y = \frac{v_L + v_R}{2}, \quad
\omega = \frac{v_R - v_L}{d}
\]

\subsection*{4.3 Net Velocity and Force}

Net forward velocity is directly average of wheel velocities:

\[
v_{\text{net}} = v_y = \frac{v_L + v_R}{2}
\]

Net force applied forward is sum of both wheels:

\[
F_{\text{net}} = F_L + F_R = 2F
\]

No strafe component (non-holonomic).



\section*{5. Grand Comparison Table}

\begin{center}
\begin{tabular}{|l|c|c|c|c|c|c|c|}
\hline
\textbf{Drive} & \textbf{Net Velocity} & \textbf{Net Force} \\
\hline
X-Drive & \( \sqrt{2} v \) & \( \sqrt{2} F \) \\
\hline
Mecanum & \( v \) & \( F \) \\
\hline
Tank & \( \frac{v_L + v_R}{2} \) & \( 2F \) \\
\hline
\end{tabular}
\end{center}



\end{document}
